\pageId{usecases}

There are a healthy number of tools which have functionality related to SnuggleTeX,
which can make it hard to decide which tool bits fits your given requirements.
Like all tools, SnuggleTeX has its relative strengths and weaknesses that you
should consider:

\begin{itemize}

\item
  SnuggleTeX was originally developed to support our
  \href[Aardvark Content Management System]{http://www.ph.ed.ac.uk/elearning/projects/aardvark/}
  in order to facilitate the conversion of fragments of LaTeX written by
  academics into XML tree branches. Another example of this type of use
  might be a kind of LaTeX-based Wiki, where SnuggleTeX could be plugged in and
  used at some point in the LaTeX to XHTML conversion pipeline.

\item
  SnuggleTeX is 100\% Java with minimal (usually no) dependencies on
  other libraries so can be easily integrated into a Java software
  development project as a library for converting LaTeX to XML.

\item
  SnuggleTeX can also be used to generate
  \href[``legacy'' web pages]{docs://legacyWebPages}
  where mathematical formulae are represented by HTML + CSS (if suitably
  simple) and/or images. (This uses the open-source
  \href[JEuclid]{http://jeuclid.sourceforge.net} library.)

\item
  SnuggleTeX was \emph{not} intended to be a standalone tool that you
  could throw complete LaTeX documents at and have them converted into web
  pages. Other tools do this type of thing very well, such as
  \href[TeX4ht]{http://www.cse.ohio-state.edu/~gurari/TeX4ht/}.

\item
  SnuggleTeX supports a pragmatic subset of LaTeX but does not include anything
  that is particularly paper- or page-specific. It also currently doesn't
  do cross-referencing or numbering as it could be argued that this is better
  done at a higher level. Other tools might therefore be a better fit for these
  types of requirements.

\item
  SnuggleTeX's parser pretends that TeX never happened and may behave slightly
  differently to what experienced LaTeX users might expect. Alternatively,
  novice LaTeX users will not notice any difference and might actually find the
  error messages provided here more helpful!

\item
  SnuggleTeX's web page outputs are highly configurable and make it relatively
  easy to create MathML-enabled pages that will work across a range of
  browser platforms. Hooks are available to customise the web page outputs to
  more exacting requirements; these require a knowledge of XSLT.

\item
  The new (and slightly experimental) features introduced in SnuggleTeX 1.1.0
  for converting LaTeX to Content MathML and Maxima were added for the
  \href[JISC MathAssess Project]{http://mathassess.ecs.soton.ac.uk/}
  in order to ``understand'' responses made by students to test questions. These
  features might have applications for similar projects.

\item
  A quick summary on the types of problem that SnuggleTeX (and related
  packages) attempt to address can be found in some
  \href[slides on conversion of Mathematical Content]{http://physics-elearning.blogspot.com/2009/10/some-slides-on-conversion-of.html}.

\end{itemize}
